\documentclass{article}

% Language setting
% Replace `english' with e.g. `spanish' to change the document language
\usepackage[english]{babel}

% Set page size and margins
% Replace `letterpaper' with `a4paper' for UK/EU standard size
\usepackage[letterpaper,top=2cm,bottom=2cm,left=3cm,right=3cm,marginparwidth=1.75cm]{geometry}

% Useful packages
\usepackage{amsmath}
\usepackage{physics}
\usepackage{graphicx}
\usepackage[colorlinks=true, allcolors=blue]{hyperref}

\title{superconducting qubits}
\author{siddhant midha}

\begin{document}
\maketitle

\begin{abstract}
Your abstract.
\end{abstract}

\section{Introduction}
DiVincenzo criteria:
\begin{itemize}
    \item \textit{scalability}. current flows in superconducting loops on an integrated circuit
    \item \textit{initialization}. cooling at cryogenic temperatures ($\sim 10$mK)
    \item \textit{coherence}. superconducting ground state!
    \item \textit{gates}. single qubit: microwave photons. two qubit: circuit QED.
    \item \textit{measurement}. EM fields.
\end{itemize}

\section{Superconductivity}
The BCS form of superconductivity manifests as a macroscopic degenerate ground state formed by \textit{pairs} of electrons. A pair looks like this:
\begin{equation}
    \sim \ket{k,\uparrow} + \ket{-k,\downarrow}
\end{equation}
These pairs, known as \textit{Cooper pairs} are bosonic excitations which result in a macroscopic occupation of the ground state. This is nothing but Bose-Einstein condensation. The ground state is described by the BCS order parameter
\begin{equation}
    \Psi = |\Psi|e^{\iota\theta}
\end{equation}
with $|\Psi|^2 = n_s$ denoting the density of the pairs.
\section{Quantum Oscillators}
Consider an oscillator described by the following differential equation
\begin{equation}
    \Ddot{Q} + \frac{1}{LC}Q = 0
\end{equation}
This dynamics is generated from the following Lagrangian
\begin{equation}
    L = \frac{1}{2}L\dot{Q}^2 - \frac{Q^2}{2C}
\end{equation}
as can be seen by a direct application of the Euler-Lagrange equations
\begin{equation}
    \frac{d}{dt}\frac{dL}{d\dot{Q}} = \frac{dL}{dQ}.
\end{equation}
The natural frequency of this oscillator is $\omega = 1/\sqrt{LC}$. The `momentum' variable here is 
\begin{equation}
    \Phi = \frac{dL}{d\dot{Q}} = L\dot{Q} 
\end{equation}
which is nothing but the magnetic flux through the inductor. These canonical variables upon quantizing satisfy
\begin{equation}
    [\hat{Q},\hat{\Phi}] = \iota\hbar 
\end{equation}
and the Hamiltonian can be written as
\begin{equation}
    \hat{H} = \frac{\hat{\Phi}^2}{2L} + \frac{\hat{Q}^2}{2C}.
\end{equation}
The ladder operators for this oscillator can be defined as
\begin{align}
    \hat{a} &= \frac{1}{\sqrt{2C\hbar\omega}}\hat{Q}  + \iota\frac{1}{\sqrt{2L\hbar\omega}}\hat{\Phi} \\
    \hat{a}^{\dagger} &= \frac{1}{\sqrt{2C\hbar\omega}}\hat{Q}  - \iota\frac{1}{\sqrt{2L\hbar\omega}}\hat{\Phi}
\end{align}
which satisfy $[\hat{a},\hat{a}^{\dagger}]=1$ as usual. What is the Hilbert space? Working in the number/charge basis, we have a set of eigenstates $\ket{n}$ for $n \geq 0$ which are eigenvalues of the number operator $\hat{N}\ket{n}=n\ket{n}$ and also of the charge operator as $\hat{Q} \propto \hat{N}$ where the constant of proportionality does not matter for now. Then, we can write the Fourier transform of basis states as
\begin{equation}
    \ket{n} = \int e^{-\iota n\phi} \ket{\phi} d\phi
\end{equation}
where $\ket{\phi}$ is the flux basis.
\section{Josephson junctions}
\begin{equation}
    \Delta V = \frac{\hbar}{2e}\frac{d(\Delta \phi)}{dt}
\end{equation}
\begin{equation}
    I = I_c\sin{\Delta \phi}
\end{equation}
\section{Quantum Circuits}
Energy in the oscillator,
\begin{equation}
    U = \int_{-\infty}^t I(t')V(t')dt' = \int_{-\infty}^t I_c\sin\phi \frac{\hbar}{2e}\dot{\phi}dt = \frac{I_c \hbar}{2e}\int_{\phi(-\infty)}^{\theta(t)}\sin\theta d\theta = -\frac{I_ce}{2\hbar} \cos\theta + c
\end{equation}
\begin{equation}
    H = \frac{Q^2}{2C} - E_J\cos\phi 
\end{equation}
with $E_J = eI_c/(2\hbar)$. Expanding the cosine $\cos\phi \approx 1-\phi^2/2$ and ignoring the constant offset,
\begin{equation}
    H = \frac{1}{2}CV^2 + \frac{1}{2}LI^2 = \frac{Q^2}{2C} + \frac{\Phi^2}{2L}
\end{equation}
with $\omega = 1/\sqrt{LC}$, $\phi = -2e\Phi/\hbar$ and $V = -LdI/dt = Q/C$.


Since now we are dealing with Cooper pairs, we have that $\hat{Q} = 2e\hat{N}$. Thus, $\hat{Q}^2/2C = (2e^2/C)\hat{N}^2 \equiv E_C\hat{N}^2$ where we define $E_C=2e^2/C$. This results in
\begin{equation}
    \boxed{H = E_C\hat{N}^2 - E_J\cos\hat{\Phi}}
\end{equation}
Expanding this out,
\begin{equation}
    H = \sum_N E_CN^2\ket{N}\bra{N} -\frac{E_J}{2}(\ket{N}\bra{N+1} + h.c.)
\end{equation}



The anharmonicity of this oscillator scales as $\sqrt{E_C/E_J}$.

Let's quantize this oscillator.

\section{trivia}
\begin{enumerate}
\item $k_BT=\hbar\omega$
\item $5$GHz $\to$ $T = 250$mK.
\item Cooper pair box $T_2^*\sim 100$ns
\item the effective dipole moment, which enters in the transition matrix elements while driving between two states, can be made much larger in superconducting qubits than atoms
\end{enumerate}

\bibliography{sample}

\end{document}