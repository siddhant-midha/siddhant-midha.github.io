\section{Introduction}
A bipartite pure state is entangled (with respect to that partition) iff it cannot be written as a pure state $\ket{\psi} = \ket{\psi_A}\otimes\ket{\psi_B}$. But, one might wish to quantify \textit{how} entangled a state is compared to another. There are carefully designed ways to do this, the simplest of which we talk about now. 


The notation of bipartite pure state entanglement is fully captured by the following theorem.
\begin{theorem}[Schmidt Decomposition]
    For any bipartite pure state $\ket{\psi} \in \H_A\otimes\H_B$, there exists orthornormal bases $\{\ket{\alpha_i}\}$ and $\{\ket{\beta_i}\}$ of $\mathcal{H}_A$, $\mathcal{H}_B$ respectively, and positive numbers $\lambda_i$ such that 
    \begin{equation}
        \ket{\psi} = \sum_{i} \sqrt{\lambda_i} \ket{\alpha_i} \otimes \ket{\beta_i}
    \end{equation}
    such that $\sum_i\lambda_i = 1$.
\end{theorem}
The values $\{\lambda_i\}$ comprise the Schmidt spectrum of $\ket{\psi}$, and characterize everything about the entanglement at that bipartition. Intuitively, if the distribution $\lambda_i$ is `well spread out,' then the entropy is high. This is quantified by the following entropic measures.

\begin{definition}[Renyi Entropies]
    For any distribution $p = \{p_i\}_i \in \mathcal{P}_n$, one defines for all $\alpha \geq 0$, $\alpha \neq 0$
    \begin{equation}
        H_\alpha(p) := \frac{1}{1-\alpha}\log\left(\sum_{i\in[n]}p_i^\alpha\right)
    \end{equation}
    For $\alpha = 1$, we define 
    \begin{eqnarray}
        H_1(p) \equiv H(p) := -\sum_{i\in[n]}p_i \log{p_i}
    \end{eqnarray}
\end{definition}
The $\alpha$-\textit{entanglement entropy} of a state $\ket{\psi} \in \H_A\otimes\H_B$ is $H_\alpha(\lambda)$, where $\lambda \equiv \{\lambda_i\}$ is the Schmidt spectrum of $\psi$.


The notion of entanglement across a bipartition is naturally then extended to mixed states.
\begin{definition}[Bipartite Separability]
    A state $\rho \in D(\H_A \otimes \H_B)$ is said to be separable if 
    \begin{equation}
        \rho = \sum_{i}p_i \rho^A_i \otimes \rho^B_i
    \end{equation}
    The set of separable states in $D(\H_A \otimes \H_B)$ is denoted $\Sep(\H_A \otimes \H_B)$.
\end{definition}
A bipartite state is said to be entanglement iff it is not separable. Also, the set $\Sep(\H_A\otimes\H_B)$ is convex, as expected. Moreover, separability implied `classicality' of the correlations, as shown by the following lemma.
\begin{lemma}[Prop.~6.6 of \cite{watrous2018theory}]
    Every separable state $\rho \in \Sep(\H_A \otimes \H_B)$ there exist $\{p_i\}_i$ and $\ket{\psi_{A(B)}}_i \in \H_{A(B)}$ for $i \in [\text{rank}(\rho)^2]$ such that
    \begin{eqnarray}
        \rho = \sum_{i}p_i \ket{\psi_{A}} \bra{\psi_{A}}_i \otimes \ket{\psi_{B}}\bra{\psi_{B}}_i 
    \end{eqnarray}
\end{lemma}
